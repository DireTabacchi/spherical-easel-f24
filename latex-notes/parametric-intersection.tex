\documentclass[11pt]{article}
\usepackage{amsmath, amsfonts}
\begin{document}
Given two parametric curves $P(s)$ and $Q(t)$ we want to find the points of intersections
between the two curves.

However, since in our app parametric curves are discretized into sample points connected by poly lines, evaluating the curves
on the sample points may fail to find the actual true solutions. As an alternative, calculating the intersection
points is implemented as a 2-step process:

\begin{enumerate}
  \item Find all pairs of discretized $(s,t)$ such that the distance between $P(s)$ and $Q(t)$ is very close (below some threshold)
  \item Use the approximate $(s,t)$ to locate the actual solution using an iterative method
\end{enumerate}

\section*{Option 1: Finding Zeros}
The first option of solving this problem can be formulated as finding the zeros of the following function
$$f(s,t) = P(s) - Q(t) = \begin{bmatrix}
    P_x(s) - Q_x(t) \\
    P_y(s) - Q_y(t) \\
    P_z(s) - Q_z(t)
  \end{bmatrix}$$
which is a function from $\mathbb{R}^2$ to $\mathbb{R}^3$.

When the $z$-component is dropped (to be verified manually after finding the $(s,t)$ solution),
the modified function becomes
$$\hat f(s,t) = P_{xy}(s) - Q_{xy}(t) = \begin{bmatrix}
    P_x(s) - Q_x(t) \\
    P_y(s) - Q_y(t)
  \end{bmatrix}$$

With the Jacobian matrix:

$$
  J =
  \begin{bmatrix}
    \frac{\partial f_x}{\partial s} & \frac{\partial f_x}{\partial t} \\
    \frac{\partial f_y}{\partial s} & \frac{\partial f_y}{\partial t} \\
  \end{bmatrix}
  =
  \begin{bmatrix}
    P'_x(s) & -Q_x'(t) \\
    P'_y(s) & -Q_y'(t)
  \end{bmatrix}
$$
the Newton's method iteration to update $(s,t)$ is given by

\begin{align*}
  \begin{bmatrix}
    s_{k+1} \\t_{k+1}
  \end{bmatrix}
   &
  =
  \begin{bmatrix}
    s_{k} \\t_{k}
  \end{bmatrix}
  -
  J^{-1} \hat f(s_k, t_k) \\
   &
  =
  \begin{bmatrix}
    s_{k} \\t_{k}
  \end{bmatrix}
  -
  \frac{1}{-P'_x(s) Q'_y(t) + P'_y(s)Q'_x(t)}
  \begin{bmatrix}
    -Q'_y(t) & Q'_x(t) \\
    -P'_y(s) & P'_x(s) \\
  \end{bmatrix} \hat f(s_k, t_k)
\end{align*}


% But since $J$ is not a square matrix, we have to compute $J^{-1}$ using a pseudo inverse
% which can be derived as follows:

% \begin{align*}
%   J
%   \begin{bmatrix}
%     s_{k+1} \\t_{k+1}
%   \end{bmatrix}
%    & =
%   J
%   \begin{bmatrix}
%     s_{k} \\t_{k}
%   \end{bmatrix}
%   -
%   f(s_k, t_k)                                                        \qquad\text{Multiply both sides by $J$} \\
%   J^t
%   J
%   \begin{bmatrix}
%     s_{k+1} \\t_{k+1}
%   \end{bmatrix}
%    & =
%   J^t
%   J
%   \begin{bmatrix}
%     s_{k} \\t_{k}
%   \end{bmatrix}
%   -
%   J^t
%   f(s_k, t_k)         \qquad\text{Multiply both sides by $J^t$ to get a square matrix}                       \\
%   \begin{bmatrix}
%     s_{k+1} \\t_{k+1}
%   \end{bmatrix}
%    & =
%   \begin{bmatrix}
%     s_{k} \\t_{k}
%   \end{bmatrix}
%   -
%   \underbrace{(J^tJ)^{-1}J^t}_\text{pseudo inverse of J} f(s_k, t_k)                                         \\
% \end{align*}

% Pseudo-inverse of Jacobian
% \begin{align*}
%   J^{-1} & =
%   \begin{pmatrix}
%     \begin{bmatrix}
%       P'(s) & Q'(t)
%     \end{bmatrix}^t
%     \begin{bmatrix}
%       P'(s) & Q'(t)
%     \end{bmatrix}
%   \end{pmatrix}^{-1}
%   \begin{bmatrix}
%     P'(s) & Q'(t)
%   \end{bmatrix}^t
%   \\
%          & =
%   \begin{pmatrix}
%     P'(s) \cdot P'(s) & P'(s) \cdot Q'(t) \\
%     P'(s) \cdot Q'(t) & Q'(t) \cdot Q'(t)
%   \end{pmatrix}^{-1}
%   \begin{bmatrix}
%     P'(s) & Q'(t)
%   \end{bmatrix}^t
% \end{align*}

% Updating $(s,t)$ becomes
% \begin{align*}
%   \begin{bmatrix}
%     s_{k+1} \\t_{k+1}
%   \end{bmatrix}
%    & =
%   \begin{bmatrix}
%     s_{k} \\t_{k}
%   \end{bmatrix}
%   -
%   \begin{pmatrix}
%     P'(s) \cdot P'(s) & P'(s) \cdot Q'(t) \\
%     P'(s) \cdot Q'(t) & Q'(t) \cdot Q'(t)
%   \end{pmatrix}^{-1}
%   \begin{bmatrix}
%     P'(s) & Q'(t)
%   \end{bmatrix}^t
%   f(s,t) \\
%    & =
%   \begin{bmatrix}
%     s_{k} \\t_{k}
%   \end{bmatrix}
%   -
%   \begin{pmatrix}
%     P'(s) \cdot P'(s) & P'(s) \cdot Q'(t) \\
%     P'(s) \cdot Q'(t) & Q'(t) \cdot Q'(t)
%   \end{pmatrix}^{-1}
%   \begin{bmatrix}
%     P'(s) \cdot f(s,t) \\
%     Q'(t) \cdot f(s,t)
%   \end{bmatrix}
% \end{align*}

\clearpage
\section*{Option 2: Minimization}
The second option is to formulate the problem as minimization of the following distance

$$d(s,t) = \sqrt{(P(s) - Q(t)) \cdot (P(s)-Q(t))}$$
a function from $\mathbb{R}^2$ to $\mathbb{R}$.

In order to have an easier function to work with, we can use the squared distance instead

$$D(s,t) = (P(s) - Q(t)) \cdot (P(s)-Q(t))$$
and we are interested in finding the zeros of $D'(s,t)$. Applying Newton's iteration again,
the Jacobian of $D'(s,t)$ is the Hessian of $D(s,t)$.

The first partial derivatives of the above function are:

\begin{align*}
  \frac{\partial D}{\partial s} & = 2(P(s) - Q(t)) \cdot P'(s) = 2(P(s) \cdot P'(s) - Q(t) \cdot P'(s)) = -2 P'(s) \cdot Q(t) \\
  \frac{\partial D}{\partial t} & = -2(P(s) - Q(t)) \cdot Q'(t) = -2(P(s)\cdot Q'(t) - Q(t)\cdot Q'(t)) = -2 P(s) \cdot Q'(t)
\end{align*}
(because $P(s)\cdot P'(s) = Q(t) \cdot Q'(t) = 0$ for spherical curves). The second partial derivatives of the above function are:

\begin{align*}
  \frac{\partial^2 D}{\partial s^2}          & = -2\left[P''(s) \cdot Q(t)\right]     \\
  \frac{\partial^2 D}{\partial t^2}          & = -2\left[P(s)  \cdot Q''(t)\right]    \\
  \frac{\partial^2 D}{\partial s \partial t} =
  \frac{\partial^2 D}{\partial t \partial s} & = -2\left[(P'(s)  \cdot  Q'(t))\right] \\
\end{align*}

and the Newton's method iteration to update $(s,t)$ is given by
$$
  \begin{bmatrix}
    s_{k+1} \\t_{k+1}
  \end{bmatrix}
  =
  \begin{bmatrix}
    s_{k} \\t_{k}
  \end{bmatrix}
  -
  \begin{bmatrix}
    \frac{\partial^2 D}{\partial s^2}          & \frac{\partial^2 D}{\partial s \partial t} \\
    \frac{\partial^2 D}{\partial t \partial s} & \frac{\partial^2 D}{\partial t^2}
  \end{bmatrix}^{-1}
  \begin{bmatrix}
    \frac{\partial D}{\partial s} \\
    \frac{\partial D}{\partial t}
  \end{bmatrix}
$$
\end{document}